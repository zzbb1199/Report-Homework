\chapter{计算机软件安全}
\question{什么是软件安全}
软件安全是在软件生命周期中,运用系统安全工程的
技术原则保证软件采取正确的措施以增强系统安全,
确保那些使系统安全性降低的错误已被消除或已被
控制在一个可接受的危险级别内

\question{软件安全保护什么?}
软件的完整性、可用性、保密性、运行安全性。

\question{软件安全}
\begin{enumerate}
	\item 软件自身安全
	\begin{enumerate}
		\item 软件自身完整性
		\item 软件自身可信性
		\item 隐蔽通道,秘密的和未公开发表的进入软件模块的
		入口
	\end{enumerate}
	\item 软件存储安全
	\begin{enumerate}
		\item 存储介质(磁盘)的可靠性
		\item 文件系统的组织结构
	\end{enumerate}
	\item 软件通信安全
	\begin{enumerate}
		\item 安全传输
		\item 加密传输
		\item 网络安全下载
		\item 完整下载	
	\end{enumerate}
	\item 软件运行安全
	\begin{enumerate}
		\item 软件运行正确性
		\item 运行日期和时间
		\item 软件补丁
	\end{enumerate}
\end{enumerate}

\question{软件安全保护机制}
\begin{enumerate}
	\item 软件防复制(存储访问技术)
	\item 软件防执行(运行控制技术)
	\item 软件防暴露(加密解密技术)
	\item 软件防篡改(完整可用技术)
\end{enumerate}

\question{软件安全性测试}
\begin{description}
	\item[安全性测试的目的] 在测试软件系统中对程序的危险防
	止和危险处理进行的测试,以验证其是否有效
	\item[安全性测试方法]
	\begin{enumerate}
		\item 功能验证
		\item 漏洞扫描
		\item 模拟攻击
	\end{enumerate}
\end{description}

