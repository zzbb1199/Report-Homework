\chapter{PKI技术}
\question{PKI 的概念及作用}
PKI是一个用\textbf{非对称密码算法}原理和技术来实现并\textbf{提供安全服务}的
具有通用性的安全基础设施。能够为所有网络应用提供采用加密
和数字签名等密码服务所需要的\textbf{密钥和证书管理}。

\question{为什么需要PKI?}
在网络、信息系统上,需要统一的、安全的认证技术

\question{提供的服务}
\begin{enumerate}
	\item 认证:PKI 通过证书进行认证,这个证书是一个可信的第三方证明的,通过它,通信双方可以安全地进行互相认证,而
	不用担心对方是假冒的。
	\item 数据保密:通过加密证书,通信双方可以协商一个密钥,而这个密钥可以作为通信加密的密钥。
	\item 完整性与不可否认:通过数字签名和可信的第三方仲裁来保证完整性和不可否认。
\end{enumerate}
\question{证书内容}
1、版本信息 2、序列号 3、签名算法 4、发行机构 5、有效期 6、拥有者 7、公开密钥 8、签名
\question{PKI的组成}
\begin{itemize}
	\item PKI 策略
	\item 软硬件系统 
	\item 注册机构(RA) 
	\item 认证中心(CA) \textbf{核心}
	\item 证书签发系统 
	\item PKI 应用 
	\item PKI 应用接口系统
\end{itemize}

\question{PKI 互通的实现途径}
\begin{enumerate}
	\item 根CA之间的交叉认证。桥接
	\item 全球性统一根CA。代理
\end{enumerate}