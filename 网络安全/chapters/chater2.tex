\chapter{密码学概论}

\question{RSA算法}\\
公钥与密钥的产生
\begin{enumerate}
	\item 随意选择两个大的质数p和q,p不等于q,计算N=pq。
	\item 根据欧拉函数,求得r = (p-1)(q-1)
	\item 选择一个小于 r 的整数 e,求得 e 关于模 r 的模反元素,命名为d。(模反元素存在,\textbf{当且仅当e与r互质})
	\begin{equation}\label{key}
	e\times d = 1\ mod\ r
	\end{equation}
	\item 将 p 和 q 的记录销毁。
\end{enumerate}
(N,e)为公钥,(N,d)为私钥。\textbf{采用公钥加密,私钥解密}\\
一个实例
\begin{gather}
 p = 3,q=11 , N = pq = 33\\
 r = (p-1)(q-1) = 2*10 = 20 \\
 ed = r * n + 1\\
 e = 3,d = 7\\
\end{gather}
所以公钥(33,3),私钥(33,7)。要传输的消息m = 24
\begin{gather}
 \text{加密}\quad m^e mod N = 24^3 \% 33 = 30\\
 \text{解密}\quad c^d mod N = 30 ^d \% 33 =24
\end{gather}
