\chapter{概论}
%\section{基础}
\vspace{-5em}
\noindent\question{信息系统的脆弱性主要体现在以下几个方面:}
\begin{enumerate}
	\item 物理因素,物理设备的自然、人为破坏。
	\item 网络因素,信息系统软件复杂度越来越高。
	\item 系统因素,信息系统软件复杂度越来越高
	\item 应用因素,不正确的操作和人为的蓄意破坏。
	\item 管理因素,管理制度、法律不健全。
\end{enumerate}
\question{信息安全的目标:}对网络中的硬件、软件和系统中的数据进行保护,不受偶然或者恶意的因素影响而遭到破坏、更改、泄漏,使系统能稳定可靠正常地运行,使信息服务不中断。

\question{信息安全关注的几点问题:}
\begin{enumerate}
	\item 密码理论与技术
	\item 安全协议理论与技术
	\item 安全体系结理论与技术
	\item 信息对抗理论与技术
	\item 网络安全与安全产品
\end{enumerate}

\question{互联网络的特点:}
\begin{enumerate}
	\item 无中心网,再生能力强
	\item 可实现移动通信、多媒体通信等多种服务
	\item 互联网一般分为外部网和内部网
	\item 互联网的用户主体是个人
\end{enumerate}
%\section{安全模型}
%\subsection{ P2DR 模型}
\question{P2DR 模型}
\begin{itemize}
	\item Policy 安全策略
	\item Protection 防护
	\item Detection  检测
	\item Response 响应
\end{itemize}
%\subsection{ PDRR 模型}
\question{PDRR模型}
\begin{itemize}
	\item Protection 防护
	\item Detection 检测
	\item Response 响应
	\item Recovery 恢复
\end{itemize}
%\section{安全体系结构}
\question{ISO开方系统互联安全体系的五类安全服务}
\begin{enumerate}
	\item 鉴别服务
	\item 访问控制服务
	\item 数据机密性服务
	\item 数据完整性服务
	\item 抵抗性
\end{enumerate}