\documentclass{article}
% 导言区
\usepackage{xeCJK}
\usepackage{graphicx}
\usepackage{float}
\usepackage{setspace}
% 文档的基本信息
\title{通信网理论基础报告}
\author{张兴锐}
\date{2018-3-10}
% 自定义
% 问题计数器
\newcounter{question}[subsection]
\renewcommand{\thequestion}{\thesection.%
	\arabic{question}}
\newcommand{\question}[1]{
	\stepcounter{question}
	\textbf{问题\thequestion \hspace{0.5em} #1}
	}

% 正文区
\begin{document}
%	\maketitle
	\tableofcontents
	 \section{引论}
	 \subsection{通信网的组成要素}
	 \question{当今典型网络的终端设备、设备线路、交换设备,主要设备的工作原理和功能}
	 \begin{description}
	 	\item[终端设备] 主要功能是把待传送的信息和在信道上传送的信号之间的相互转换。
		\begin{itemize}
			\item 发送传感起来感受信息和接受传感器将信号恢复称能被利用的信息
			\item 能处理信号的设备使之能与信道匹配
			\item 能产生和辨别网内所需的信令或协议,以便相互联系和应答
		\end{itemize}
		\item[设备线路] 传输线路,踏实电磁波传输的路径。通常分为无界传播和导引传播。
		\begin{enumerate}
			\item\textbf{无线信道} 传输通道主要自由空间,需要发射机、发射天线、接受天线和接收机。
			\begin{enumerate}
				\item \textbf{长波线路}300$kHz$以下。沿地面尤其是海岸的传播损耗小。一般用于航海系统中
				\item \textbf{短波线路}3$MHz$-30$MHz$。传播损耗已较大,但利用地球上空电离层反射,可进行远距离通信。
				\item \textbf{微波线路}\ 作为通信网的信道的主要方式是中继线路或接力线路。
			\end{enumerate}
			\item \textbf{有线信道} 电磁波是沿道题传播的,而且通常是能构成直流通路,适宜与基带传输。包括\textbf{架空明线、平衡电缆、同轴电缆、波导传输。},除了有线线路还需要增加\textbf{增音器和均衡器}。
		\end{enumerate}
		\item[交换设备] 终端设备和信道是构成通信系统的必要设施,除此之外,还需要交换设备
		\begin{enumerate}
			\item \textbf{电路交换}
			\item \textbf{分组交换}\ 分组交换在网路资源利用上比电路交换方式好,但总要引入一定延时,所以对实时要求高的如电话通信不利。
			\item \textbf{多址接入}\ 上两者都需要讲交换信息传送到一个交换点或转接站。引入多址接入可使哥哥用户直接接送到线路上去。
		\end{enumerate}
	 \end{description}
	 \question{电话网与计算机通信网的不同}
	 \begin{enumerate}
	 	\item 传统电话网使用电路交换方式,而计算机通信网多使用分组交换方式或虚电路方式。
	 	\item 电话网对交换的实时性要求高,但对准确率要求相对较低。计算机通信网则对实时性要求相对低,准确率要求高。
	 	\item 电话网传输速率相对计算机通信网普遍较低。
	 \end{enumerate}
	 \question{通信网的约定的概念,电话网的约定、因特网的约定}
		\begin{enumerate}
			\item \textbf{通信网的约定}\ 网内使用的一种``语言'',用他们来协调网的运行,达到互通、互控和互换的目的。
			\item \textbf{电话网的约定}\ 电话信令
			\item \textbf{因特网的约定}\ 计算机通信协议
		\end{enumerate}
	 \question{通信网的质量标准及传输标准}
	 	\begin{enumerate}
	 		\item \textbf{质量标准}\ 质量决定于信道的比特误码率
			\begin{enumerate}
				\item 接续质量,受网资源的容量和可靠性限制,主要靠增加网资源来提高
				\item 信息质量,受终端额信道的失真和噪声等限制,因信息类型的不同而不同。
			\end{enumerate}
			\item \textbf{传输标准}\ 规定了信道接口的一系列参数。
	 	\end{enumerate}
\end{document}
