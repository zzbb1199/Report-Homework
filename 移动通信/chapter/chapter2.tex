\chapter{移动信道的传播特性于模型}
\section{无线电波传播方式
}
移动通信使用\textbf{甚高频(VHF)}和\textbf{特高频(UHF)}等频段传输.
\begin{description}
	\item[VHF:] 1--10m,30--300MHZ
	\item[UHF:] 10--100CM,300MHZ--3GHZ
\end{description}
当频率f > 30 MHZ时,传播通路主要有:直射波、反射波、地表面波。
\subsection{直射波}
直射波传播,可按自由空间传播来考虑。
\begin{description}
	\item[条件:] 自由空间传播是指天线周围为无
	限大真空时的电波传播
	\item[现象] 不发生反射、折射、绕射、散射
	和吸收等现象,但电波经过一段路径传播之
	后,能量仍会衰减,这是由电磁波能量扩散而引起的传播损耗
	(弥散损耗或称为自由空间传播损耗)。
	\begin{eqnarray}
	L_{fs}(dB) = 32.44 + 20lg_d(km)+20lg_f(MHZ)
	\end{eqnarray}
\end{description}
传播接受能量排序:直射>(反射、绕射)>散射。
\subsection{反射波}
物体尺寸比传输波长大得多(如地面,墙面),则容易发生镜面反射。
\subsection{绕射波}
尖利边缘(山丘)。

余隙:障碍物顶点P到直射线TR的距离,称为菲涅尔余隙。阻挡时为负(即当障碍物高于TR线时),反之为正。\textbf{附加损耗可通过查表得到},该损耗和相对余隙$x/x1$有关,其中\(x1\)第一菲涅尔半径,x为余隙长度。
\begin{eqnarray}
x_1 = \sqrt{\frac{\lambda d_1d_2}{d_1+d_2}}
\end{eqnarray}
\subsection{散射}
比传输波长小的多的物体(粗
糙表面、不规则物体),并且单位体积内阻挡体的个数很堵的情况。
\subsection{反射、绕射和散射的关系}
\begin{tabular}{|c|c|}
	\hline
		&阻挡体	\\
		\hline
		反射	&	比传输波长大的多的物体(地
		面、墙面)	\\
				\hline
		绕射	&	尖利边缘(山丘)\\
				\hline
		散射	&	比传输波长小的多的物体(粗
		糙表面、不规则物体)	\\
				\hline
\end{tabular}

\section{移动信道的特性
}













