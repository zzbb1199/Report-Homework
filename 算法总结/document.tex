\documentclass[a4paper]{article}
% 导言
\usepackage{graphicx}
\usepackage{float}
\usepackage{xeCJK}
\usepackage{xcolor} 
\usepackage{fontspec}
\usepackage{amsmath}
\usepackage{array}
\usepackage{listings}
\usepackage{hypertoc}
\usepackage{hyperref}
\definecolor{mygreen}{rgb}{0,0.6,0}
\definecolor{mygray}{rgb}{0.5,0.5,0.5}
\definecolor{mymauve}{rgb}{0.58,0,0.82}
\setlength{\arrayrulewidth}{1mm}
\setlength{\tabcolsep}{15pt}
\renewcommand{\arraystretch}{2.0}
\lstset{ %
	backgroundcolor=\color{white},   % choose the background color
	basicstyle=\footnotesize\ttfamily,        % size of fonts used for the code
	columns=fullflexible,
	breaklines=true,                 % automatic line breaking only at whitespace
	captionpos=b,                    % sets the caption-position to bottom
	tabsize=4,
	commentstyle=\color{mygreen},    % comment style
	escapeinside={\%*}{*)},          % if you want to add LaTeX within your code
	keywordstyle=\color{blue},       % keyword style
	stringstyle=\color{mymauve}\ttfamily,     % string literal style
	frame=single,
	rulesepcolor=\color{red!20!green!20!blue!20},
	% identifierstyle=\color{red},
	language=java,
}


% 文档标题
\title{算法知识总结}
\author{张兴锐}
\begin{document}
	\maketitle
	\tableofcontents
	
	\section{暴力法}
	暴力穷--\textbf{填空题}常用。
	\begin{itemize}
		\item 暴力破解中的实用性原则
		\item 枚举法,从前往后推。
		\item 逆向解法,从后往前推。
	\end{itemize}
	\subsection{枚举法:年龄问题}
	美国数学家维纳(N.Wiener)智力早熟,11岁就上了大学。他曾在1935~1936年应邀来中国清华大学讲学。
	一次,他参加某个重要会议,年轻的脸孔引人注目。于是有人询问他的年龄,他回答说:“我年龄的立方是个4位数。我年龄的4次方是个6位数。这10个数字正好包含了从0到9这10个数字,每个都恰好出现1次。”请你推算一下,他当时到底有多年轻?
	\lstinputlisting{codes/AgeProblem.java}
	\subsection{水仙花数}
	这里不给出具体代码,对于这个问题既可以逆向思维也可以正向推。比如求解三位数的水仙花数字:可以从100-999进行循环取出各个位数进行比较,也可以嵌套撒能循环,i1=0-9,i2=0-9,i3=0-9。这个三个数字生成的数字的是否满足条件。两种情况时间复杂度也是相同的,都要循环1000次。
	\section{递归解法}
	递归解法用处广泛,不够总体上分为两种:
	\begin{enumerate}
		\item \textbf{最终数量型},对于这种问题,主要是考虑问题递归的前后联系关系,尽量不要使用到公共结构和全局变量,因为不需要输出最终结果。\textbf{思想常常是当前状态下可有几种选择,有几种就+上几种进行迭代,当然也需要设计好出口条件}
		\begin{enumerate}
			\item 求解全排列的个数,或者求解组合个数。
		\end{enumerate}
		\item \textbf{最终结果型},或者在上述情况下无法方便求出的情况下,可以考虑这种方法,常常需要使用到公共数据结构和全局变量,\textbf{回溯法特别常用}。
		\begin{enumerate}
			\item 具有边界条件的问题,如裁剪问题
			\item 输出全排列,利用交换思想或者利用填数思想。
			\item 输出具有重复数字的组合。
			\item n皇后问题,等等。
		\end{enumerate}
	\end{enumerate}
	\subsection{类型一:求解最终数量}
	\subsubsection{观光台售票问题}
	公园票价为5角。假设每位游客只持有两种币值的货币:5角、1元。
	再假设持有5角的有m人,持有1元的有n人。
	由于特殊情况,开始的时候,售票员没有零钱可找。
	我们想知道这m+n名游客以什么样的顺序购票则可以顺利完成购票过程。
	显然,m < n的时候,无论如何都不能完成;
	m>=n的时候,有些情况也不行。比如,第一个购票的乘客就持有1元。
	请计算出这m+n名游客所有可能顺利完成购票的不同情况的组合数目。
	注意:只关心5角和1元交替出现的次序的不同排列,持有同样币值的两名游客交换位置并不算做一种新的情况来计数。
	\lstinputlisting{codes/ChangeProblem.java}
	\subsubsection{甲虫出站问题}
	\lstinputlisting{codes/OutStack.java}
	\subsection{类型二:最终结果型}
	\subsubsection{求解全排列}
	\textbf{暴力破解中常用到,所以特别指出}
	\lstinputlisting{codes/Test.java}
	\subsubsection{具有重复字母的排列问题}
	给定一个具有重复数字的字符集合,如“AABBCCD”,从中取4个数字,有多少种解法。
	\lstinputlisting{codes/TakeNumProblem.java}
	\subsubsection{裁剪纸问题}
	\lstinputlisting{codes/Probleam4.java}
	
	\section{算法中的数学问题}
	\subsection{进制的巧妙使用}
	\textbf{在题目中有明显的进制迭代关系,可以考虑使用本方法}.
	\subsubsection{捐赠问题}
	地产大亨Q先生临终的遗愿是:拿出100万元给X社区的居民抽奖,以稍慰藉心中愧疚。
	麻烦的是,他有个很奇怪的要求:
	1. 100万元必须被正好分成若干份(不能剩余)。
	
	每份必须是7的若干次方元。
	比如:1元, 7元,49元,343元,...
	
	2. 相同金额的份数不能超过5份。
	
	3.在满足上述要求的情况下,分成的份数越多越好!
	
	请你帮忙计算一下,最多可以分为多少份?\\
	\textit{对于本题,1,7,49,343是明显的7的次方,可以采用7进制进行求解。}
	\lstinputlisting{codes/FenQian.java}
	
	\subsubsection{天平称重问题-改造的3进制问题}
	用天平称重时,我们希望用尽可能少的砝码组合称出尽可能多的重量。
	如果只有5个砝码,重量分别是1,3,9,27,81
	则它们可以组合称出1到121之间任意整数重量(砝码允许放在左右两个盘中)。
	本题目要求编程实现:对用户给定的重量,给出砝码组合方案。
	例如:
	
	用户输入:
	
	5
	
	程序输出:
	
	9-3-1
	
	用户输入:
	
	19
	
	程序输出:
	
	27-9+1
	
	要求程序输出的组合总是大数在前小数在后。
	
	可以假设用户的输入的数字符合范围1~121。
	\lstinputlisting{codes/TianPingProblem.java}
	
	\subsection{最小最大公约数}
	\textbf{最大公约数}
	欧几里得定理:辗转相处法。$gcd(a,b) = gcd(b,a\%b)直到b=0$
	\textbf{最小公倍数}
	$a*b/gcd(a,b)$
	
	\subsection{无误差有理数计算}
	
	如果求 1/2 + 1/3 + 1/4 + 1/5 + 1/6 + .... + 1/100 = ?
	要求绝对精确,不能有误差。
	\lstinputlisting{codes/FuDianJingDu.java}
	
	\subsection{筛选法求解素数}
	
	第1个素数是2,第2个素数是3,...
	求第100002(十万零二)个素数
	
	其基本思路是数字排成串,从首位向后面扫描,将所有能够约束的给删除。另外\textit{java中linklist的add操作更快,arraylist的get操作更快}
	\lstinputlisting{codes/SuShuFind.java}
	
	\subsection{不定方程的求解}
	\textbf{方法一:暴力求解} \\
	通过暴力求解,答题上有几个变量就会有几层循环,但是通常可以通过一定的变形将问题的循环层数减少,例如求解
	\begin{displaymath}
		ax+by=c
	\end{displaymath}
	可以通过两层循环进行试探,也可以变形为
	\begin{displaymath}
		x = (c-by)/a
	\end{displaymath}
	循环变为一层,即循环y,当上式子右边能够整除$a$时,得到结果。\\
	\textbf{方法二:拓展欧几里得}\\
	拓展欧几里得定理:

	\begin{gather*}
		ax_1+by_1=gcd(a,b) \\
		bx_2+(a\%b)y-2=gcd(b,a\%b)		\\
		\text{结束条件},a\%b = 0
	\end{gather*}
	所以可以设计递归进行求解。
	\lstinputlisting{codes/ExpandOJLiDe.java}
	
	\section{博弈论}
	博弈论,常有两种解法:
	\begin{enumerate}
		\item 固定套路,采用暴力解
		\item 尼姆堆定理
	\end{enumerate}
	\subsection{暴力解法}
	
	\subsubsection{伪代码如下:}
	\lstinputlisting{codes/boyilun_weidaima.java}
	\subsubsection{取球问题}
	今盒里有n个小球,A、B两人轮流从盒中取球。
	每个人都可以看到另一个人取了多少个,也可以看到盒中还剩下多少个。
	两人都很聪明,不会做出错误的判断。
	
	每个人从盒子中取出的球的数目必须是:1,3,7或者8个。
	轮到某一方取球时不能弃权!
	A先取球,然后双方交替取球,直到取完。
	
	被迫拿到最后一个球的一方为负方(输方)
	
	编程确定出在双方都不判断失误的情况下,对于特定的初始球数,A是否能赢?
	\lstinputlisting{codes/TakeBall.java}
	
	\subsubsection{尼姆堆}
	有3堆硬币,分别是3,4,5
	
	二人轮流取硬币。
	
	每人每次只能从某一堆上取任意数量。
	
	不能弃权。
	
	取到最后一枚硬币的为赢家。
	
	求先取硬币一方有无必胜的招法。
	
	\textbf{尼姆定理:将所有堆的数量进行异或操作,如果得到的结果非全零(非平衡状态)则先手赢,得到的是全零则先手输}\\
	\textbf{所有无偏的二人游戏都可以用尼姆定理进行求解,关键在于如何将问题转化为尼姆堆}
	\lstinputlisting{codes/NiMuDui.java}
	
	\section{随机算法}
	如果题目要求的是一种可行解,这种方法非常适合,较优解这种方式也可用。推荐使用\textbf{蒙特卡洛和模拟退火},稍微好实现一些。
	\subsection{蒙特卡洛算法}
	题目:给定4个数,看它是否能通过+ - * / 四种运算符得到24点。\\
	这个题有两个值得学习的地方:
	\begin{enumerate}
		\item 逆波兰表达式计算
		\item 抛异常跳出大循环,原来一直在想exit()函数的问题,可以用这种方式来实现。
		\item 使用try-catch-finally,语句,即使在try或者catch语句中执行了return,finally语句也会被执行。
	\end{enumerate}
	\lstinputlisting{codes/CalCu24.java}
	\subsection{模拟退火}
	\subsubsection{伪代码}
	\lstinputlisting{codes/sa_weidaima.java}
	\subsubsection{求解$\sin x在(-\pi,\pi)$之间的最小值}
	\lstinputlisting{codes/SA.java}
	
	\section{动态规划}
	
	垒骰子
	
	赌圣atm晚年迷恋上了垒骰子,就是把骰子一个垒在另一个上边,不能歪歪扭扭,要垒成方柱体。
	经过长期观察,atm 发现了稳定骰子的奥秘:有些数字的面贴着会互相排斥!
	我们先来规范一下骰子:1 的对面是 4,2 的对面是 5,3 的对面是 6。
	假设有 m 组互斥现象,每组中的那两个数字的面紧贴在一起,骰子就不能稳定的垒起来。 atm想计算一下有多少种不同的可能的垒骰子方式。
	两种垒骰子方式相同,当且仅当这两种方式中对应高度的骰子的对应数字的朝向都相同。
	由于方案数可能过多,请输出模 $ 10^9 + 7 $ 的结果。
	
	不要小看了 atm 的骰子数量哦~
	
	「输入格式」
	第一行两个整数 n m
	n表示骰子数目
	接下来 m 行,每行两个整数 a b ,表示 a 和 b 不能紧贴在一起。
	
	「输出格式」
	一行一个数,表示答案模$  10^9 + 7 $ 的结果。
	
	「样例输入」
	2 1
	1 2

	「样例输出」
	544
	
	本题注意两点:
	\begin{itemize}
		\item 数据量大,暴力解肯定超过范围,所以使用DP。
		\item 大规模余数,将余数因子写成变量,且在用科学表达式时要先进行强制转换并赋值给变量。不要直接使用。
	\end{itemize}
	\lstinputlisting{codes/DP1.java}
	
	
	\section{其他技巧}
	\subsection{树}
	\subsubsection{二叉树}
	\begin{enumerate}
		\item 二叉排序树:通过二分进行建树,中序遍历即可
		\item 哈弗曼编码树:最小无重复前缀编码方式
		\item 区间树:
		\item 线段树:经常对某个区间的某个值进行修改,对某个区间进行查询。
	\end{enumerate}
	\subsubsection{哈弗曼编码树}
	\lstinputlisting{codes/HaffuManTree.java}
	\subsubsection{线段树}
	基本操作:\begin{itemize}
		\item 修改区间值
		\item 查询区间特征值(视情况而定) 
	\end{itemize}
	\textbf{注意事项:树的申请空间为原数据的4倍}
	\lstinputlisting{codes/SegmentTree.java}
	\subsubsection{树dp}
	没有上司的舞会。\textbf{可以学到:}
	\begin{enumerate}
		\item 如何建树。
		\item 树dp。从下往上更新。
	\end{enumerate}
	\lstinputlisting{codes/NoShangSi.java}
	\subsection{正则表达式}
	\subsubsection{关键词}
	\begin{table}[H]
		\begin{tabular}{c|c}
			\hline
			符号	&	含义	\\
			\hline
			$\backslash b$ &	匹配一个字的边界  \\
			$\backslash B$ &	非字符边界	\\
			$\backslash d$ &	数字	\\
			$\backslash D$ &	非数字	\\
			$\backslash s$ &	空白字符\\
			$\backslash S$ &	非空白字符\\
			$\backslash w$ &	单词字符\\
			$\backslash W$ &	非单词字符\\
			\hline
		\end{tabular}
	\end{table}
	\subsubsection{去叠词}
	
	\begin{lstlisting}
	public class Test {
	public static void main(String[] args) {
		String test = "22223131413718623872193132121111";
		System.out.println(test.replaceAll("(.)\\1{2,}", "$1"));
		System.out.println(test.replaceAll("(.)\\1+", "$1"));
	}
	}
	\end{lstlisting}

\end{document}